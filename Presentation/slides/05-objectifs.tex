%==============================================================================
% SLIDE 05 - OBJECTIFS
%==============================================================================

% Slide de transition
\sectionSlide{03}{Objectifs}

% Slide de contenu
\begin{frame}
\frametitle{Objectifs du Projet}
\framesubtitle{Ce que nous voulons accomplir}
\customFooter

\vspace{0.2cm}

\begin{columns}[T]
%----------------------------------------------------------------------
% COLONNE GAUCHE - OBJECTIFS PRINCIPAUX
%----------------------------------------------------------------------
\begin{column}{0.48\textwidth}

    % Titre
    \begin{tikzpicture}
        \node[
            rectangle,
            rounded corners=4pt,
            fill=primary,
            text=white,
            inner xsep=10pt,inner ysep=6pt,
            font=\small\bfseries
        ] {\faIcon{bullseye}\ \ Objectifs Principaux};
    \end{tikzpicture}
    
    \vspace{0.5cm}
    
    % Liste d'objectifs
    \begin{tikzpicture}[node distance=0.5cm]
        \node[anchor=west] (o1) at (0,0) {
            \textcolor{secondary}{\faIcon{check-circle}}\ \ 
            Développer une plateforme \textbf{multi-plateforme}
        };
        
        \node[anchor=west,below=of o1.west,anchor=west] (o2) {
            \textcolor{secondary}{\faIcon{check-circle}}\ \ 
            Intégrer un modèle d'\textbf{Intelligence Artificielle}
        };
        
        \node[anchor=west,below=of o2.west,anchor=west] (o3) {
            \textcolor{secondary}{\faIcon{check-circle}}\ \ 
            Créer une \textbf{communauté} de partage
        };
        
        \node[anchor=west,below=of o3.west,anchor=west] (o4) {
            \textcolor{secondary}{\faIcon{check-circle}}\ \ 
            Offrir une interface \textbf{intuitive} et moderne
        };
        
        \node[anchor=west,below=of o4.west,anchor=west] (o5) {
            \textcolor{secondary}{\faIcon{check-circle}}\ \ 
            Assurer la \textbf{sécurité} des données
        };
    \end{tikzpicture}

\end{column}

%----------------------------------------------------------------------
% COLONNE DROITE - FONCTIONNALITÉS
%----------------------------------------------------------------------
\begin{column}{0.48\textwidth}

    % Titre
    \begin{tikzpicture}
        \node[
            rectangle,
            rounded corners=4pt,
            fill=secondary,
            text=white,
            inner xsep=10pt,inner ysep=6pt,
            font=\small\bfseries
        ] {\faIcon{tasks}\ \ Fonctionnalités Cibles};
    \end{tikzpicture}
    
    \vspace{0.5cm}
    
    % Grille de fonctionnalités
    \begin{tikzpicture}[
        feat/.style={
            rectangle,
            rounded corners=6pt,
            fill=secondary,
            fill opacity=0.1,
            text opacity=1,
            minimum width=6cm,
            minimum height=0.7cm,
            inner sep=6pt,
            font=\small
        }
    ]
        \node[feat] (f1) at (0,0) {
            \textcolor{secondary}{\faIcon{lock}}\ \ 
            Authentification sécurisée JWT
        };
        
        \node[feat,below=0.2cm of f1] (f2) {
            \textcolor{secondary}{\faIcon{calculator}}\ \ 
            Simulations de poutres
        };
        
        \node[feat,below=0.2cm of f2] (f3) {
            \textcolor{secondary}{\faIcon{bolt}}\ \ 
            Calculs IA en temps réel
        };
        
        \node[feat,below=0.2cm of f3] (f4) {
            \textcolor{secondary}{\faIcon{history}}\ \ 
            Historique des simulations
        };
        
        \node[feat,below=0.2cm of f4] (f5) {
            \textcolor{secondary}{\faIcon{share-alt}}\ \ 
            Partage communautaire
        };
        
        \node[feat,below=0.2cm of f5] (f6) {
            \textcolor{secondary}{\faIcon{cube}}\ \ 
            Visualisation 3D interactive
        };
    \end{tikzpicture}

\end{column}

\end{columns}

\end{frame}
