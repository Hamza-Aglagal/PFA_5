%==============================================================================
% SLIDE 12 - MODÈLE D'INTELLIGENCE ARTIFICIELLE
%==============================================================================

\begin{frame}
\frametitle{Modèle d'Intelligence Artificielle}
\framesubtitle{Deep Learning pour la Simulation Structurelle}
\customFooter

\vspace{0.2cm}

\begin{columns}[T]
%----------------------------------------------------------------------
% COLONNE GAUCHE - ARCHITECTURE RÉSEAU
%----------------------------------------------------------------------
\begin{column}{0.52\textwidth}

    % Titre
    \begin{tikzpicture}
        \node[
            rectangle,
            rounded corners=4pt,
            fill=accent,
            text=white,
            inner xsep=10pt,inner ysep=5pt,
            font=\small\bfseries
        ] {\faIcon{network-wired}\ \ Architecture du Réseau de Neurones};
    \end{tikzpicture}
    
    \vspace{0.4cm}
    
    % Schéma du réseau
    \begin{center}
    \begin{tikzpicture}[
        scale=0.75,
        transform shape,
        neuron/.style={circle,draw=primary,fill=primary!20,minimum size=0.5cm,line width=0.5pt},
        hidden/.style={circle,draw=accent,fill=accent!30,minimum size=0.5cm,line width=0.5pt},
        output/.style={circle,draw=secondary,fill=secondary!30,minimum size=0.5cm,line width=0.5pt}
    ]
        % Couche d'entrée
        \node[font=\tiny\bfseries,text=primary] at (0,3) {Entrées};
        \foreach \i in {1,...,5} {
            \node[neuron] (i\i) at (0,{2.5-\i*0.6}) {};
        }
        
        % Couche cachée 1
        \node[font=\tiny\bfseries,text=accent] at (2,3) {Couche 1};
        \foreach \i in {1,...,6} {
            \node[hidden] (h1\i) at (2,{2.8-\i*0.55}) {};
        }
        
        % Couche cachée 2
        \node[font=\tiny\bfseries,text=accent] at (4,3) {Couche 2};
        \foreach \i in {1,...,6} {
            \node[hidden] (h2\i) at (4,{2.8-\i*0.55}) {};
        }
        
        % Couche de sortie
        \node[font=\tiny\bfseries,text=secondary] at (6,3) {Sorties};
        \foreach \i in {1,...,3} {
            \node[output] (o\i) at (6,{1.5-\i*0.7}) {};
        }
        
        % Connexions simplifiées
        \foreach \i in {1,...,5} {
            \foreach \j in {1,...,6} {
                \draw[gray400,opacity=0.3,line width=0.3pt] (i\i) -- (h1\j);
            }
        }
        \foreach \i in {1,...,6} {
            \foreach \j in {1,...,6} {
                \draw[gray400,opacity=0.3,line width=0.3pt] (h1\i) -- (h2\j);
            }
        }
        \foreach \i in {1,...,6} {
            \foreach \j in {1,...,3} {
                \draw[gray400,opacity=0.3,line width=0.3pt] (h2\i) -- (o\j);
            }
        }
        
        % Labels entrées
        \node[font=\tiny,text=gray600,left=0.1cm of i1] {Longueur};
        \node[font=\tiny,text=gray600,left=0.1cm of i2] {Largeur};
        \node[font=\tiny,text=gray600,left=0.1cm of i3] {Hauteur};
        \node[font=\tiny,text=gray600,left=0.1cm of i4] {Charge};
        \node[font=\tiny,text=gray600,left=0.1cm of i5] {Matériau};
        
        % Labels sorties
        \node[font=\tiny,text=gray600,right=0.1cm of o1] {Déflexion};
        \node[font=\tiny,text=gray600,right=0.1cm of o2] {Contrainte};
        \node[font=\tiny,text=gray600,right=0.1cm of o3] {Sécurité};
    \end{tikzpicture}
    \end{center}

\end{column}

%----------------------------------------------------------------------
% COLONNE DROITE - PERFORMANCES
%----------------------------------------------------------------------
\begin{column}{0.45\textwidth}

    % Performance
    \begin{tikzpicture}
        \node[
            rectangle,
            rounded corners=4pt,
            fill=secondary,
            text=white,
            inner xsep=10pt,inner ysep=5pt,
            font=\small\bfseries
        ] {\faIcon{chart-line}\ \ Performances};
    \end{tikzpicture}
    
    \vspace{0.3cm}
    
    \begin{tikzpicture}
        \node[
            rectangle,
            rounded corners=8pt,
            fill=secondary!10,
            draw=secondary,
            draw opacity=0.3,
            minimum width=5.8cm,
            inner sep=10pt
        ] {
            \begin{minipage}{5.2cm}
                \textbf{Dataset :} 10,000 simulations FEM\\[0.15cm]
                \textbf{Précision :} $>$ 95\%\\[0.15cm]
                \textbf{Temps :} $<$ 100ms / prédiction\\[0.15cm]
                \textbf{Architecture :} MLP 5-128-64-3
            \end{minipage}
        };
    \end{tikzpicture}
    
    \vspace{0.4cm}
    
    % Entrées
    \begin{tikzpicture}
        \node[
            rectangle,
            rounded corners=4pt,
            fill=primary,
            text=white,
            inner xsep=10pt,inner ysep=5pt,
            font=\small\bfseries
        ] {\faIcon{sign-in-alt}\ \ Paramètres d'Entrée};
    \end{tikzpicture}
    
    \vspace{0.2cm}
    
    {\small
    \begin{itemize}
        \item[\textcolor{primary}{\faIcon{ruler}}] Dimensions de la poutre
        \item[\textcolor{primary}{\faIcon{weight-hanging}}] Type et magnitude de charge
        \item[\textcolor{primary}{\faIcon{industry}}] Type de matériau
        \item[\textcolor{primary}{\faIcon{cog}}] Module d'élasticité
    \end{itemize}
    }
    
    \vspace{0.3cm}
    
    % Sorties
    \begin{tikzpicture}
        \node[
            rectangle,
            rounded corners=4pt,
            fill=accent,
            text=white,
            inner xsep=10pt,inner ysep=5pt,
            font=\small\bfseries
        ] {\faIcon{sign-out-alt}\ \ Résultats Prédits};
    \end{tikzpicture}
    
    \vspace{0.2cm}
    
    {\small
    \begin{itemize}
        \item[\textcolor{accent}{\faIcon{compress-alt}}] Déflexion maximale
        \item[\textcolor{accent}{\faIcon{exclamation-triangle}}] Contrainte maximale
        \item[\textcolor{accent}{\faIcon{shield-alt}}] Facteur de sécurité
    \end{itemize}
    }

\end{column}

\end{columns}

\end{frame}
