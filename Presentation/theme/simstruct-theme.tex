%==============================================================================
% Thème SIMSTRUCT - Design Professionnel et Créatif
%==============================================================================

\usepackage[utf8]{inputenc}
\usepackage[T1]{fontenc}
\usepackage[french]{babel}
\usepackage{lmodern}           % Police Latin Modern (supporte grandes tailles)
\usepackage{graphicx}
\usepackage{tikz}
\usepackage{fontawesome5}
\usepackage{booktabs}
\usepackage{array}
\usepackage{xcolor}
\usepackage{hyperref}
\usepackage{calc}
\usepackage{eso-pic}
\usepackage{transparent}
\usepackage{fix-cm}            % Correction pour les grandes tailles de police

\usetikzlibrary{shapes.geometric,positioning,calc,shadows.blur,decorations.pathreplacing}

%------------------------------------------------------------------------------
% PALETTE DE COULEURS PROFESSIONNELLE
%------------------------------------------------------------------------------
\definecolor{primary}{RGB}{37, 99, 235}        % Bleu principal
\definecolor{primaryDark}{RGB}{29, 78, 216}    % Bleu foncé
\definecolor{primaryLight}{RGB}{96, 165, 250}  % Bleu clair
\definecolor{secondary}{RGB}{16, 185, 129}     % Vert succès
\definecolor{accent}{RGB}{245, 158, 11}        % Orange accent
\definecolor{danger}{RGB}{239, 68, 68}         % Rouge erreur
\definecolor{dark}{RGB}{30, 41, 59}            % Texte foncé
\definecolor{gray600}{RGB}{71, 85, 105}        % Gris moyen
\definecolor{gray400}{RGB}{148, 163, 184}      % Gris clair
\definecolor{gray100}{RGB}{241, 245, 249}      % Fond très clair
\definecolor{white}{RGB}{255, 255, 255}        % Blanc

%------------------------------------------------------------------------------
% CONFIGURATION BEAMER DE BASE
%------------------------------------------------------------------------------
\usetheme{default}
\usecolortheme{default}
\setbeamertemplate{navigation symbols}{}
\setbeamertemplate{footline}{}
\setbeamertemplate{headline}{}

% Couleurs globales
\setbeamercolor{background canvas}{bg=white}
\setbeamercolor{normal text}{fg=dark}
\setbeamercolor{structure}{fg=primary}
\setbeamercolor{itemize item}{fg=primary}
\setbeamercolor{itemize subitem}{fg=primaryLight}

% Polices
\setbeamerfont{title}{size=\fontsize{42}{48}\selectfont,series=\bfseries}
\setbeamerfont{frametitle}{size=\Large,series=\bfseries}
\setbeamerfont{framesubtitle}{size=\normalsize}

% Items
\setbeamertemplate{itemize item}{\textcolor{primary}{\rule[0.5ex]{0.6ex}{0.6ex}}}
\setbeamertemplate{itemize subitem}{\textcolor{primaryLight}{\rule[0.5ex]{0.5ex}{0.5ex}}}

%------------------------------------------------------------------------------
% COMMANDES UTILITAIRES
%------------------------------------------------------------------------------

% Icône avec cercle
\newcommand{\iconCircle}[2][primary]{%
    \tikz[baseline=-0.5ex]{
        \node[circle,fill=#1,minimum size=1.8em,inner sep=0pt] {\textcolor{white}{\small#2}};
    }%
}

% Badge numéroté
\newcommand{\badge}[2][primary]{%
    \tikz[baseline=-0.5ex]{
        \node[circle,fill=#1,minimum size=2em,inner sep=0pt,font=\bfseries\small] {\textcolor{white}{#2}};
    }%
}

% Carte avec ombre
\newcommand{\card}[3][primary]{%
    \begin{tikzpicture}
        \node[
            rectangle,
            rounded corners=8pt,
            fill=white,
            draw=#1,
            line width=1pt,
            minimum width=#2,
            minimum height=1cm,
            inner sep=12pt,
            drop shadow={shadow xshift=1pt, shadow yshift=-2pt, opacity=0.15}
        ] {#3};
    \end{tikzpicture}%
}

% Titre de section coloré
\newcommand{\sectionTitle}[2]{%
    \begin{tikzpicture}
        \node[
            rectangle,
            rounded corners=4pt,
            fill=#1,
            inner sep=10pt,
            font=\bfseries
        ] {\textcolor{white}{#2}};
    \end{tikzpicture}%
}

% Check et Cross marks
\newcommand{\cmark}{\textcolor{secondary}{\faIcon{check-circle}}}
\newcommand{\xmark}{\textcolor{danger}{\faIcon{times-circle}}}

%------------------------------------------------------------------------------
% TEMPLATE POUR SLIDES DE SECTION
%------------------------------------------------------------------------------
\newcommand{\sectionSlide}[2]{%
{
\setbeamertemplate{footline}{}
\begin{frame}[plain]
\begin{tikzpicture}[remember picture,overlay]
    % Fond dégradé
    \fill[primary] (current page.south west) rectangle (current page.north east);
    
    % Cercles décoratifs
    \fill[primaryDark,opacity=0.5] ([xshift=-3cm,yshift=2cm]current page.south west) circle (6cm);
    \fill[primaryLight,opacity=0.2] ([xshift=4cm,yshift=-1cm]current page.north east) circle (5cm);
    \fill[white,opacity=0.08] ([xshift=-2cm,yshift=-2cm]current page.north west) circle (3cm);
    
    % Ligne décorative
    \draw[white,line width=3pt] ([yshift=-0.5cm]current page.center) ++(-3cm,0) -- ++(6cm,0);
    
    % Numéro de section
    \node[white,font=\fontsize{72}{80}\selectfont\bfseries,opacity=0.3] 
        at ([yshift=2cm]current page.center) {#1};
    
    % Titre de section
    \node[white,font=\fontsize{36}{42}\selectfont\bfseries] 
        at ([yshift=-0.5cm]current page.center) {#2};
        
\end{tikzpicture}
\end{frame}
}
}

%------------------------------------------------------------------------------
% TEMPLATE POUR FRAME TITRE PERSONNALISÉ
%------------------------------------------------------------------------------
\setbeamertemplate{frametitle}{
    \vspace{0.6cm}
    \begin{beamercolorbox}[wd=\paperwidth]{frametitle}
        \hspace{0.6cm}
        \begin{tikzpicture}[baseline]
            % Barre latérale colorée
            \fill[primary] (0,0) rectangle (0.15cm,0.8cm);
            % Titre
            \node[anchor=west,font=\Large\bfseries,text=dark] at (0.4cm,0.4cm) {\insertframetitle};
        \end{tikzpicture}
        
        \ifx\insertframesubtitle\empty\else
            \hspace{1.15cm}\textcolor{gray600}{\small\insertframesubtitle}
        \fi
    \end{beamercolorbox}
}

%------------------------------------------------------------------------------
% FOOTER PERSONNALISÉ
%------------------------------------------------------------------------------
\newcommand{\customFooter}{
    \begin{tikzpicture}[remember picture,overlay]
        % Ligne de séparation
        \draw[gray400,line width=0.5pt] 
            ([xshift=0.8cm,yshift=0.6cm]current page.south west) -- 
            ([xshift=-0.8cm,yshift=0.6cm]current page.south east);
        
        % Auteur
        \node[anchor=west,font=\tiny,text=gray600] 
            at ([xshift=0.8cm,yshift=0.3cm]current page.south west) {\insertshortauthor};
        
        % Titre
        \node[anchor=center,font=\tiny\bfseries,text=primary] 
            at ([yshift=0.3cm]current page.south) {SIMSTRUCT};
        
        % Numéro de page
        \node[anchor=east,font=\tiny,text=gray600] 
            at ([xshift=-0.8cm,yshift=0.3cm]current page.south east) 
            {\insertframenumber\ /\ \inserttotalframenumber};
    \end{tikzpicture}
}

%------------------------------------------------------------------------------
% ENVIRONNEMENT POUR PLACEHOLDER D'IMAGE
%------------------------------------------------------------------------------
\newcommand{\imagePlaceholder}[3][13cm]{%
    \begin{tikzpicture}
        \node[
            rectangle,
            rounded corners=12pt,
            fill=gray100,
            draw=primary,
            line width=2pt,
            minimum width=#1,
            minimum height=#2,
            inner sep=0pt
        ] (box) {};
        
        \node[text=gray400,font=\fontsize{36}{42}\selectfont] at (box.center) {\faIcon{image}};
        \node[text=gray600,font=\small,below=0.3cm of box.center,text width=8cm,align=center] {#3};
    \end{tikzpicture}%
}
